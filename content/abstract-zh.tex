% !Mode:: "TeX:UTF-8"

% 中英文摘要
\begin{cabstract}
	分布式存储系统作为一种可以存储海量数据并提供便捷数据访问的服务在大数据时代被广泛应用。由于系统庞大而导致数据丢失的情况时有发生,系统一般采用
	纠删码的方式来进行数据存储并冗余。但是纠删码存在数据修复计算开销大,网络负载高,修复速度低等问题。本文针对以上问题,对纠删码的预先修复和混合纠删码修复技术
	进行了深入研究,主要研究与贡献如下:
	
	(1)针对纠删码修复速度慢而导致系统可靠性降低问题,本文在预先重建与迁移修复的基础上,改进预先修复算法使其适应快速变化的网络环境,提出
	了划分迁移集和重建集算法SMSRS,并引入多级中继传输概念提出了改进的多级传输调度算法ISA加速修复任务数据的传输。实验结果表明,本文提出的
	划分算法和调度技术较传统方法修复时间降低了7.2\%$\sim$13.2\%。

	(2)针对传统单一纠删码修复技术的退化读高延迟问题,本文在传统混合纠删码LRC\&HH码的基础上,提出了一种可感知数据热度的负载动态自适应
	的混合纠删码数据修复技术。该技术根据实际存储系统I/O特点,定义了读写负载模式和修复负载模式,通过建模分析编码分配规则和动态队列划分冷热
	数据算法提出负载自适应策略,并结合编码切换算法进而提升系统数据修复速度。实验结果表明,本文提出的负载自适应策略,相比于现有的LRC\&HH码随机
	切换策略,修复时间降低了6.8\%$\sim$18.7\%。

	(3)基于上述理论研究成果,设计实现了一个基于混合纠删码的容错存储原型系统。该系统含有多种混合纠删码方案,包括
	HACFS、EC-Fusion以及LRC\&HH,支持相应的扩展接口,支持修复调度中的重建与迁移的融合。系统包含以下模块:存储中心架构,节点故障模型,混合
	纠删码策略,节点放置策略,可靠性度量指标,事件处理模式。系统可接受含存储中心架构、混合纠删码策略和冗余设置等参数,并且可以定量地分析
	各种纠删码方案之间的可靠性量化指标。实验结果表明,基于预先修复的混合纠删码方案与传统单一纠删码相比可以提升系统可靠性,其中PDL参数降低了30\%$\sim$35\%,
	NOMDL降低了40\%$\sim$42.5\%,BR降低了35.6\%$\sim$37.8\%。
	
	\vskip 21bp
	\noindent
	{\heiti\zihao{-4} 关键词:}
	分布式存储,
	预先修复,
	混合纠删码,
	数据修复
	
	\begin{flushright}
		% 作~~~~~~~~者:叶冬冬
		
		% 指导老师:郑朝晖
		作~~~~~~~~者:***
		
		指导老师:***
		
	\end{flushright}
\end{cabstract}



% !Mode:: "TeX:UTF-8"

\begin{eabstract}
	Distributed storage systems are widely used in the era of big data as a service that can store 
	large amounts of data and provide convenient data access, but data loss occurs due to the large system, 
	and the system generally uses the erasure code for data storage and redundancy. However, 
	there are problems such as high computational overhead of data repair, high network load and low repair speed. 
	In this paper, we address the above problems and conduct an in-depth study on the pre-repair and hybrid erasure code repair techniques for erasure codes.
	The main research and contributions are as follows:

	(1) In order to address the problem of slow repair speed of erasure codes which leads to the reduction of system reliability, 
	this paper improves the pre-repair algorithm to adapt to the fast changing network environment based on the pre-repair and migration 
	repair, proposes the division of migration set and reconstruction set algorithm SMSRS, and introduces the concept of multi-stage 
	relay transmission and proposes an improved multi-stage transmission scheduling algorithm ISA to speed up the transmission of 
	repair task data. The experimental results show that the proposed partitioning algorithm and scheduling technique reduce the repair time by 
	7.2\%$\sim$13.2\% compared with the traditional method.

	(2) To address the degraded read-high latency problem of traditional single-code repair technology, 
	this paper proposes a load dynamic adaptive hybrid censored data repair technology that can 
	sense data heat based on the traditional hybrid censored LRC\&HH codes. The technique defines 
	the read/write load mode and repair load mode according to the I/O characteristics of the 
	actual storage system, proposes a load adaptive strategy by modeling and analyzing the code 
	allocation rules and dynamic queueing algorithm to divide hot and cold data, and combines the 
	code switching algorithm to improve the system data repair speed. The experimental results show 
	that the proposed load adaptive strategy reduces the repair time by 6.8\%$\sim$18.7\% compared with the existing LRC\&HH code random switching strategy.

	(3) Based on the above theoretical research results, 
	a prototype system of fault-tolerant storage based on hybrid censoring codes is designed 
	and implemented. The system contains multiple hybrid censoring schemes, including HACFS, 
	EC-Fusion and LRC\&HH, and supports the corresponding extension interfaces and the fusion 
	of reconstruction and migration in repair scheduling. The system includes the following modules: 
	storage center architecture, node failure model, hybrid code correction policy, node placement policy, reliability metrics, and event
	processing model. The system can accept parameters containing storage center architecture, hybrid coding strategy and redundancy settings, 
	and can quantitatively analyze the reliability quantifiers among various coding schemes. 
	The experimental results show that the hybrid pre-repair-based censoring scheme can improve the system reliability 
	compared with the traditional single censoring scheme, in which the PDL parameters are reduced by 30\%$\sim$35\%, NOMDL by 40\%$\sim$42.5\%, and BR by 35.6\%$\sim$37.8\%.


	\vskip 21bp
	\noindent
	{\bf\zihao{-4} Key words: }
	Distributed Storage,
	Predictive repair,
	Mixed erasure Code,
	Data repair
\end{eabstract}

\begin{flushright}
	Written by Dongdong Ye
	
	Supervised by Zhaohui Zheng
	% Written by ***
	
	% Supervised by ***
\end{flushright}

% !Mode:: "TeX:UTF-8"

% 中英文摘要
\begin{cabstract}
	句法分析任务是句子理解的重要中间过程之一.
	其中,概率估计一直是句法分析领域的一个核心问题.
	然而,无论是神经网络方法还是深度学习时代以前的方法,采用基于全局概率模型的句法分析工作都非常少,主要的原因在于树形条件随机场(TreeCRF)推断的高复杂度.
	在本文中,我们提出将TreeCRF应用到依存句法和成分句法这两个主要的句法分析任务.
	为了解决TreeCRF的低效问题,关键的想法是批次化树结构的推断算法,并且用基于自动求导的反向传播代替Outside算法.
	目前句法模型被不断简化,采用局部损失目标是当前句法分析方法的一个趋势,我们则进一步在一阶TreeCRF的基础上采用了高阶拓展.
	高阶TreeCRF进一步增加了算法复杂度,为此,我们还提出利用基于平均场变分推断的近似推断算法代替精确推断的TreeCRF方法,从而增加了解析效率.
	
	\vskip 21bp
	{\heiti\zihao{-4} 关键词:}
	句法分析,
	依存句法分析,
	成分句法分析,
	树形条件随机场,
	变分推断
	
	\begin{flushright}
		作~~~~~~~~者:叶冬冬
		
		指导老师:郑朝晖
		
	\end{flushright}
\end{cabstract}



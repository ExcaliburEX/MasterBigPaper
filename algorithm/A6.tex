\begin{algorithm}[htpb]
	\begin{algorithmic}[1]
		% \newlength{\commentindent}
		\setlength{\commentindent}{.3\textwidth}
		\setlength{\algorithmicindent}{1.5em}
		\renewcommand{\algorithmiccomment}[1]{\unskip\hfill\makebox[\commentindent][l]{$\rhd$~#1}\par}
		\LetLtxMacro{\oldalgorithmic}{\algorithmic}
		\renewcommand{\algorithmic}[1][0]{
			\oldalgorithmic[#1]
			\renewcommand{\ALC@com}[1]{
				\IFnum\pdfstrcmp{##1}{default}=0\ELSE\algorithmiccomment{##1}\fi}%
		}
		\REQUIRE{Queue1,Queue2 and $\eta$}
		% \ENSURE{$\textbf{NewW}$}
        \FOR{\text{Queue2头的每个插入请求}}
        \IF{flag $\neq $ Fast Code and (writes/recoveries) $< \eta$}
        \STATE set flag = Fast Code;
        \STATE convert to Fast Code;
        \ENDIF
        \ENDFOR
        \FOR{\text{Queue1头的每个插入请求}}
        \IF{flag $\neq $ Compact Code and (writes/recoveries) $\geqslant  \eta$}
        \STATE set flag = Compact Code;
        \STATE convert to Compact Code;
        \ENDIF
        \ENDFOR
        \FOR{\text{Queue2尾的每个弹出请求}}
        \IF{flag == Fast Code }
        \STATE set flag = Compact Code;
        \STATE convert back to Compact Code;
        \ENDIF
        \ENDFOR
	\end{algorithmic}
	\caption{容错存储原型系统动态自适应划分算法}
	\label{alg:5-1}
\end{algorithm}